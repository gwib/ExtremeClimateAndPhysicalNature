\section{}

\subsection{New Terms}
\textbf{Mass balance:} The mass balance of a glacier or ice sheet is the net balance between the mass gained by snow deposition, and the loss of mass by melting (either at the glacier surface or under the floating ice shelves or ice tongues) and calving (production of icebergs). A negative mass balance means that a glacier is losing mass, and, for grounded glaciers and ice sheets, this mass loss directly contributes to sea level rise (the melting of floating ice shelves and ice tongues does not contribute to sea level rise, because of the lower density of ice as compared to water, which determines the floating portion of the ice). \cite{massBalance}


\subsection{Running the Model 1: Cooling Climate}
For a colder climate our model ice sheet becomes bigger, primarily due to less surface melting (accumulation area is larger). The eustatic sea level lowers as well due to the presence of Northern Hemisphere ice sheets, so that the ice sheet can gradually expand horizontally.


