\section{Conclusion and Prospect} 
Traffic Management Technologies and especially Car-to-X systems are emerging and growing technologies that are certainly going to gain huge importance in the future. This technology is very sensitive to attacks and error and faults may have even life-endangering outcome. \\
Figure\ref{fig:automotive} sums up the possible goals of attacks in the car-to-car communication, such as apps, the network, the physical interface and the different attack actions, such as tracking and interception, spamming, or manipulation. \\

\begin{figure}[h!]
        \centering
        \includegraphics[scale=0.5]{graphics/automotive.png}
        \caption{E2E automotive security \cite{automotive}}
        \label{fig:automotive}
\end{figure}

As already mentioned, the scope of these systems is growing, imposing more and more challenges. On the other hand, the challenges seem to be well-identified and there is done a lot of research on securing this technology. \\
A lot of physical installations need to be installed to for example ensure the distribution of fresh vehicle keys based on location, but not only for security, but also to give more accuracy to positioning. However, these measures are quite costly, in addition to creating new vulnerabilities. \\
The traditional security measures may not be sufficient. That is where Machine Learning comes in. Machine Learning can be used to both identify and classify anomalies (see Figure \ref{fig:automotive}). It can also used to learn patterns to filter out abnormal messages or messages that seem to be malicious or that seem to have an abnormal sender. \\
Finally, especially concerning privacy, policies for ensuring the protection of users' (drivers') data need to be established, as it is not necessarily the main interest of the platform developers.