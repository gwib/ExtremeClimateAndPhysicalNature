\section{Q1: State of the art in Climate Change - Sea Ice}

\subsection{Sea Ice and Climate Change}
Sea Ice is the ice that forms, grows and melts in the ocean. It is floating on the surface and its movement is determined by winds and currents. The average thickness of sea ice is 3 metres; however, the range is between mm-range and 10 metres thickness. Its extent expands and contracts remarkably from winter to summer. When the ice sheet is formed, it does not take in all the salt, which then affects the movement of ocean waters, if it happens in a large scale.
Compared to other Earth surfaces, e.g. the Ocean, Sea Ice has a very high albedo (1), reflecting the Sun’s radiation almost entirely.
Sea Ice is interesting looking at its extent, thickness, concentration, drift and snow thickness above the ice (influencing the albedo). To measure these components, one needs to carry out both remote  sensing measurements and in situ (ground-based) measurements, setting up ice camps (what they used to do a lot earlier), or by using icebreaker ships. The remote sensing “methods” are visible, infrared, active and passive microwave.
Looking at the Sea Ice extent, it has declined rapidly within the past 30 years, showing a minimizing of 13.2 per cent per decade.
This leads to the following changes in the climate and especially the marine system:
\begin{itemize}
    \item Fresher ocean water
    \item More evaporation
    \item Warming oceans (due to decreased albedo)
    \item More extreme events, e.g. storms, increased wave height
    \item Coastal erosion
    \item More acidic water due to higher CO2 uptake
	\item Interruption of the thermohaline circulation
\end{itemize}
As these impacts are not only global, it is necessary to try to understand (model) and predict the development of the Sea Ice. To achieve that, there are two different types of model: one is based on statistical data and the other one is a dynamic model, simulating the interaction between the ocean, atmosphere, the land surface and ice. The models are verified and their performance is measured by using given data to predict for a time that we have data for. However, it will be more and more complicated to create reliable models, as the shrinkage of the multi-year ice increases the year-to-year variability, which is hard to account for.\\
The changes of Sea Ice do also imply effects to the anthroposphere and ecosystems:
With the Sea Ice Mass shrinking, some indigenous traditional lifestyles cannot be practised anymore, such as hunting on the sea ice. Coastal erosion does threaten structures. Some ice adapted species may lose their habitats which will lead to a changing range of the arctic species and changing diets. Some positive effects might be cheaper transport in East-Asia, Europe and North America.




\subsection{Feedback from fellow students}
Well-structured Kept the time Nice introduction (new for most of us, good to give definition), caught the attention Part about how measure using Remote Sensing difficult to understand (Galina has a background that we do not have)
